\documentclass[aspectratio=169]{beamer}

%% Layout and Style %%
% \usetheme{Berlin}

\usepackage{xcolor}
\usepackage{hyperref}
\hypersetup{
  colorlinks=false,
  urlcolor=blue,
  pdfauthor={Severen Redwood},
}
\usepackage{floatrow}

%% Typography %%
\usepackage{fontspec}
\usepackage{microtype}

%% Maths %%
\usepackage{mathtools}
\usepackage{amssymb}
\usepackage{amsthm}
\usepackage{unicode-math}
\setmathfont{TeX Gyre Pagella Math}

% Basic sets of numbers.
\newcommand*{\N}{\mathbb{N}}
\newcommand*{\Z}{\mathbb{Z}}
\newcommand*{\Q}{\mathbb{Q}}
\newcommand*{\R}{\mathbb{R}}
\newcommand*{\C}{\mathbb{C}}

\newcommand*{\F}{\mathbb{F}}

% Cardinality and order.
\DeclarePairedDelimiter{\card}{\lvert}{\rvert}
\DeclarePairedDelimiter{\ord}{\lvert}{\rvert}

%% Language %%
\usepackage{csquotes}
\usepackage{polyglossia}
\setdefaultlanguage[variant=newzealand, ordinalmonthday]{english}

% PDF Metadata
\hypersetup{
  pdftitle={Algebra and the AES Algorithm}
}

\title{Algebra and the AES Algorithm}
\author{Severen Redwood \\ {\small Supervisor: Robert Culling}}
\institute{University of Canterbury}

\begin{document}

\begin{frame}
  \titlepage
\end{frame}

\begin{frame}{What is the AES algorithm?}
The AES algorithm is a \emph{cipher} --- an algorithm for encrypting and decrypting data.
\end{frame}

\begin{frame}{What is it used for?}
  AES is now the most widely used cipher in use. \pause
  
  Some notable applications of AES are:
  \begin{itemize}[<+->]
    \item the transport layer security (TLS) protocol
    \item the Wi-Fi encryption standard (IEEE 802.11i)
    \item password managers such as LastPass, Bitwarden, 1Password
    \item chat applications such as Messenger, WhatsApp, and Skype
  \end{itemize}
\end{frame}

\begin{frame}{Why should mathematicians care?}
  There are two reasons:
  \begin{itemize}[<+->]
    \item Ciphers use mathematics such as number theory and algebra to perform their
          operations.
    \item The only way to know \emph{for sure} that a cipher is secure is to
          \emph{mathematically} prove it.
  \end{itemize}

  \pause

  The design of ciphers is called \emph{cryptography} and the analysis of their security
  is called \emph{cryptanalysis}.
\end{frame}

\begin{frame}{Where are finite fields involved?}
The AES algorithm is a \emph{block} cipher. \pause

Input data such as
\[ 11101000~01010000~11010011~00000010~00100000~01001100~00001100~10010101 \]
\pause is segmented into "blocks" of a certain length \pause (for example, 2 bytes each)
\[
  \overbrace{11101000~01010000}^A~
  \overbrace{11010011~00000010}^B~
  \overbrace{00100000~01001100}^C~
  \overbrace{00001100~10010101}^D.
\]
\pause The algorithm then operates on each block individually, ultimately combining the
processed blocks to produce a final result.
\end{frame}

\begin{frame}{Where are finite fields involved?}
  Within each block, we map each byte to an element in the finite field \(\F_{2^8}\),
  whose elements are polynomials of degree \(7\) or less with coefficients \(0\) or
  \(1\). \pause

  For example,
  \begin{align*}
    11010011 &\mapsto x^7 + x^6 + x^4 + x + 1, \\
    11101000 &\mapsto x^7 + x^6 + x^5 + x^3.
  \end{align*}
\end{frame}

% TODO: More examples, less gory details about groups/rings/fields  

\begin{frame}{What is a finite field?}
  A finite field is a finite set of objects that we can add, subtract, multiply, and
  divide as we do with real numbers. \pause

  For the AES algorithm, we care about two fields: \(\F_2 = \{0, 1\}\) and \(\F_{2^8}\).
  \pause

  In \(\F_2\):
  \begin{columns}[c]
    \begin{column}{.2\textwidth}
    \begin{tabular}{c|cc}
      + & 0 & 1 \\
      \cline{1-3}
      0 & 0 & 1 \\
      1 & 1 & 0 \\
    \end{tabular}
    \end{column}

    \begin{column}{.2\textwidth}
    \begin{tabular}{c|cc}
      \times & 0 & 1 \\
      \cline{1-3}
      0 & 0 & 0 \\
      1 & 0 & 1 \\
    \end{tabular}
    \end{column}
  \end{columns}
  \pause

  In \(\F_{2^8}\), addition is performed as with ordinary polynomials, but the
  coefficients are in \(\F_2\).
\end{frame}

\begin{frame}{Multiplication in \(\F_{2^8}\)}
  Unlike addition, multiplication in \(\F_{2^8}\) is somewhat special. \pause

  For two polynomials \(f\) and \(g\), we have that \(\deg(fg) = \deg(f) + \deg(g)\).
  Therefore, \(fg\) can result in polynomials with degree \emph{greater} than \(7\)!
  \pause

  To account for this, we define the product of \(f\) and \(g\) in \(\F_{2^8}\) to be
  \[ fg \bmod{p}, \]
  where \(p\) is an \emph{irreducible} polynomial of degree 8.
\end{frame}

\begin{frame}{Multiplication in \(\F_{2^8}\)}
  For example, let \(f(x) = x^6 + x\), \(g(x) = x^3 + 1\), and \(p(x) = x^8 +
  x^4 + x^3 + x + 1\).
  \pause
  Then,
  \[ f(x)g(x) = (x^6 + x)(x^3 + 1) = x^9 + x^6 + x^4 + x, \]
  which has degree \(9\). \pause So we reduce it using \(p\):
  \[ f(x)g(x) \bmod{p} = x^9 + x^6 + x^4 + x \bmod{p} = x^6 + x^5 + x^2 \]
\end{frame}

\begin{frame}{Thanks :)}
  Thanks to:
  \begin{itemize}
    \item Rob, for both this opportunity and many fun and insightful conversations;
    \item The audience, for listening.
  \end{itemize}
\end{frame}

\end{document}
